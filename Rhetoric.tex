% Options for packages loaded elsewhere
\PassOptionsToPackage{unicode}{hyperref}
\PassOptionsToPackage{hyphens}{url}
\PassOptionsToPackage{dvipsnames,svgnames,x11names}{xcolor}
%
\documentclass[
  letterpaper,
  DIV=11,
  numbers=noendperiod]{scrartcl}

\usepackage{amsmath,amssymb}
\usepackage{iftex}
\ifPDFTeX
  \usepackage[T1]{fontenc}
  \usepackage[utf8]{inputenc}
  \usepackage{textcomp} % provide euro and other symbols
\else % if luatex or xetex
  \usepackage{unicode-math}
  \defaultfontfeatures{Scale=MatchLowercase}
  \defaultfontfeatures[\rmfamily]{Ligatures=TeX,Scale=1}
\fi
\usepackage{lmodern}
\ifPDFTeX\else  
    % xetex/luatex font selection
\fi
% Use upquote if available, for straight quotes in verbatim environments
\IfFileExists{upquote.sty}{\usepackage{upquote}}{}
\IfFileExists{microtype.sty}{% use microtype if available
  \usepackage[]{microtype}
  \UseMicrotypeSet[protrusion]{basicmath} % disable protrusion for tt fonts
}{}
\makeatletter
\@ifundefined{KOMAClassName}{% if non-KOMA class
  \IfFileExists{parskip.sty}{%
    \usepackage{parskip}
  }{% else
    \setlength{\parindent}{0pt}
    \setlength{\parskip}{6pt plus 2pt minus 1pt}}
}{% if KOMA class
  \KOMAoptions{parskip=half}}
\makeatother
\usepackage{xcolor}
\setlength{\emergencystretch}{3em} % prevent overfull lines
\setcounter{secnumdepth}{-\maxdimen} % remove section numbering
% Make \paragraph and \subparagraph free-standing
\makeatletter
\ifx\paragraph\undefined\else
  \let\oldparagraph\paragraph
  \renewcommand{\paragraph}{
    \@ifstar
      \xxxParagraphStar
      \xxxParagraphNoStar
  }
  \newcommand{\xxxParagraphStar}[1]{\oldparagraph*{#1}\mbox{}}
  \newcommand{\xxxParagraphNoStar}[1]{\oldparagraph{#1}\mbox{}}
\fi
\ifx\subparagraph\undefined\else
  \let\oldsubparagraph\subparagraph
  \renewcommand{\subparagraph}{
    \@ifstar
      \xxxSubParagraphStar
      \xxxSubParagraphNoStar
  }
  \newcommand{\xxxSubParagraphStar}[1]{\oldsubparagraph*{#1}\mbox{}}
  \newcommand{\xxxSubParagraphNoStar}[1]{\oldsubparagraph{#1}\mbox{}}
\fi
\makeatother


\providecommand{\tightlist}{%
  \setlength{\itemsep}{0pt}\setlength{\parskip}{0pt}}\usepackage{longtable,booktabs,array}
\usepackage{calc} % for calculating minipage widths
% Correct order of tables after \paragraph or \subparagraph
\usepackage{etoolbox}
\makeatletter
\patchcmd\longtable{\par}{\if@noskipsec\mbox{}\fi\par}{}{}
\makeatother
% Allow footnotes in longtable head/foot
\IfFileExists{footnotehyper.sty}{\usepackage{footnotehyper}}{\usepackage{footnote}}
\makesavenoteenv{longtable}
\usepackage{graphicx}
\makeatletter
\newsavebox\pandoc@box
\newcommand*\pandocbounded[1]{% scales image to fit in text height/width
  \sbox\pandoc@box{#1}%
  \Gscale@div\@tempa{\textheight}{\dimexpr\ht\pandoc@box+\dp\pandoc@box\relax}%
  \Gscale@div\@tempb{\linewidth}{\wd\pandoc@box}%
  \ifdim\@tempb\p@<\@tempa\p@\let\@tempa\@tempb\fi% select the smaller of both
  \ifdim\@tempa\p@<\p@\scalebox{\@tempa}{\usebox\pandoc@box}%
  \else\usebox{\pandoc@box}%
  \fi%
}
% Set default figure placement to htbp
\def\fps@figure{htbp}
\makeatother
% definitions for citeproc citations
\NewDocumentCommand\citeproctext{}{}
\NewDocumentCommand\citeproc{mm}{%
  \begingroup\def\citeproctext{#2}\cite{#1}\endgroup}
\makeatletter
 % allow citations to break across lines
 \let\@cite@ofmt\@firstofone
 % avoid brackets around text for \cite:
 \def\@biblabel#1{}
 \def\@cite#1#2{{#1\if@tempswa , #2\fi}}
\makeatother
\newlength{\cslhangindent}
\setlength{\cslhangindent}{1.5em}
\newlength{\csllabelwidth}
\setlength{\csllabelwidth}{3em}
\newenvironment{CSLReferences}[2] % #1 hanging-indent, #2 entry-spacing
 {\begin{list}{}{%
  \setlength{\itemindent}{0pt}
  \setlength{\leftmargin}{0pt}
  \setlength{\parsep}{0pt}
  % turn on hanging indent if param 1 is 1
  \ifodd #1
   \setlength{\leftmargin}{\cslhangindent}
   \setlength{\itemindent}{-1\cslhangindent}
  \fi
  % set entry spacing
  \setlength{\itemsep}{#2\baselineskip}}}
 {\end{list}}
\usepackage{calc}
\newcommand{\CSLBlock}[1]{\hfill\break\parbox[t]{\linewidth}{\strut\ignorespaces#1\strut}}
\newcommand{\CSLLeftMargin}[1]{\parbox[t]{\csllabelwidth}{\strut#1\strut}}
\newcommand{\CSLRightInline}[1]{\parbox[t]{\linewidth - \csllabelwidth}{\strut#1\strut}}
\newcommand{\CSLIndent}[1]{\hspace{\cslhangindent}#1}

\KOMAoption{captions}{tableheading}
\makeatletter
\@ifpackageloaded{caption}{}{\usepackage{caption}}
\AtBeginDocument{%
\ifdefined\contentsname
  \renewcommand*\contentsname{Table of contents}
\else
  \newcommand\contentsname{Table of contents}
\fi
\ifdefined\listfigurename
  \renewcommand*\listfigurename{List of Figures}
\else
  \newcommand\listfigurename{List of Figures}
\fi
\ifdefined\listtablename
  \renewcommand*\listtablename{List of Tables}
\else
  \newcommand\listtablename{List of Tables}
\fi
\ifdefined\figurename
  \renewcommand*\figurename{Figure}
\else
  \newcommand\figurename{Figure}
\fi
\ifdefined\tablename
  \renewcommand*\tablename{Table}
\else
  \newcommand\tablename{Table}
\fi
}
\@ifpackageloaded{float}{}{\usepackage{float}}
\floatstyle{ruled}
\@ifundefined{c@chapter}{\newfloat{codelisting}{h}{lop}}{\newfloat{codelisting}{h}{lop}[chapter]}
\floatname{codelisting}{Listing}
\newcommand*\listoflistings{\listof{codelisting}{List of Listings}}
\makeatother
\makeatletter
\makeatother
\makeatletter
\@ifpackageloaded{caption}{}{\usepackage{caption}}
\@ifpackageloaded{subcaption}{}{\usepackage{subcaption}}
\makeatother

\usepackage{bookmark}

\IfFileExists{xurl.sty}{\usepackage{xurl}}{} % add URL line breaks if available
\urlstyle{same} % disable monospaced font for URLs
\hypersetup{
  pdftitle={Rhetorical Analysis},
  colorlinks=true,
  linkcolor={blue},
  filecolor={Maroon},
  citecolor={Blue},
  urlcolor={Blue},
  pdfcreator={LaTeX via pandoc}}


\title{Rhetorical Analysis}
\author{}
\date{}

\begin{document}
\maketitle


\section{Intro to use of Rhetoric}\label{intro-to-use-of-rhetoric}

Inácio, Rodríguez-Álvarez, and Gayoso-Diz's article ``Statistical
Evaluation of Medical Tests'' displays a variety of rhetorical
strategies. Implementing this rhetoric helps to bridge the gap between
complex methodology and clinical
application.@inacioStatisticalEvaluationMedical2021 Writing for the
\emph{Annual Review of Statistics and Its Application}, the authors
construct their argument through balancing mathmatical rigor and
practical accessibility. The combination of these strategies makes the
article comprhendable to both statisticians and medical researchers.

\subsection{Rhetorical Structure}\label{rhetorical-structure}

The authors establish credibility through their organizational
framework, and move systematically from foundational concepts to
advanced applications. They begin with binary tests before addressing
continuous measures which demonstrates an increasing complexity. This
ensures sensitivity---readers encounter familiar diagnostic concepts
(positive/negative) before presenting the complex threshold-based
classifications Inácio, Rodríguez-Álvarez, and Gayoso-Diz (2021,
44--45). This approach reflects awareness of their dual audience:
statisticians seeking methodological depth and medical practitioners
requiring practical guidance.

The integration of the HOMA-IR dataset as an example also shows
effective applied rhetoric. Rather than presenting abstract theory, the
authors relate each statistical concept in the clinical question of
predicting cardio-metabolic risk Inácio, Rodríguez-Álvarez, and
Gayoso-Diz (2021, 43). This method helps to communicate an in-depth
methods review as an accessible case study. The decision to reveal
findings incrementally, then by gender, and finally incorporating
age-effects mirrors the investigative process researchers follow, making
the methodology feel intuitive rather than imposed.

\subsection{Visual and Mathematical
Communication}\label{visual-and-mathematical-communication}

The authors use visual arguments to effectively make complex claims
accessible. Figure 2 uses distributions to show how overlap determines
diagnostic accuracy, making the relationship more apparent without
needing a lengthy explanation. Figure 3 takes a different approach by
showing four estimation methods producing similar results, which makes
the argument for a certain increased applicability without stating it
directly Inácio, Rodríguez-Álvarez, and Gayoso-Diz (2021, 52). These
visual techniques work well for showing patterns, but they convince
readers that methods work without explaining why they should be trusted
over alternatives. In contrast, the dense mathematical equations (like
Equations 3, 11, and 17) cancreate barriers for non-specialist readers
who must either trust the authors or reject the claims entirely.

\section{Informative as Persuasive}\label{informative-as-persuasive}

The article's most distinctive rhetorical technique is using assertion
through authoritative voice. Most of the claims highlighted in this
paper were stated as facts, and did not contain language meant to
convince the reader. Claims like ``ROC curves measure the amount of
separation between distributions'' are stated as facts rather than
defended as positions Inácio, Rodríguez-Álvarez, and Gayoso-Diz (2021,
46). The lack of defensive argumentation implies the claims need no
defense, which actually strengthens them. The effectiveness of this
strategy would depend on what the prior notions were surrounding the
topic and the overall purpose of this paper: was it to convince or to
persuade?

\subsection{Effectiveness Assessment}\label{effectiveness-assessment}

The article's rhetorical techniques are effective in serving readers
already familiar with ROC curves. The ordering of the information builds
understanding, visual arguments demonstrate patterns, and authoritative
assertion builds validity. The lack of comparative arguments (why ROC
curves over other approaches?) and defensive warrants (what makes these
assumptions valid?) limits persuasive. Overall, the article succeeded
more through its rhetoric, rather than its argumentative claims.

\phantomsection\label{refs}
\begin{CSLReferences}{1}{0}
\bibitem[\citeproctext]{ref-inacioStatisticalEvaluationMedical2021}
Inácio, Vanda, María Xosé Rodríguez-Álvarez, and Pilar Gayoso-Diz. 2021.
{``Statistical {Evaluation} of {Medical Tests}.''} \emph{Annual Review
of Statistics and Its Application} 8 (Volume 8, 2021): 41--67.
\url{https://doi.org/10.1146/annurev-statistics-040720-022432}.

\end{CSLReferences}




\end{document}
